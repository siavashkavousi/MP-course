\documentclass[12pt,a4paper]{article}

\usepackage{graphicx}
\usepackage{subcaption}
\usepackage{array}
\usepackage{adjustbox}
\usepackage{tablefootnote}
\usepackage{amsfonts}
\usepackage{amssymb}
\usepackage{yfonts}

\usepackage{xepersian}

\settextfont{Adobe Arabic}

\begin{document}
\begin{persian}
\begin{center}
	\thispagestyle{empty}
	\centerline{\includegraphics[height=5cm]{logo.png}}
	\vspace{20mm}
	\textbf{\Huge{پیش گزارش و گزارش آزمایش پنجم}}\\
	\vspace{10mm}
	\textit{\huge{سیاوش کاوسی 9231048 \\ آرش تارافر 9131034}}\\
	\vspace{10mm}
	\textit{\huge{آقای حیدری}}\\
\end{center}	

	\newpage
	\section{پیش گزارش}
	\subsection{مقدمه و هدف آزمایش}
		هدف آزمایش کار با درگاه ها، سرکشی یک پایه از یک درگاه، تولید تاخیر نرم افزاری، کار با زمان سنج نگهبان و کار با رله و بازر در میکروکنترلر ATmega32 است
	\subsection{لیست قطعات بکار رفته در آزمایش}
		میکروکنترلر ATmega32، چند مقاومت، کلید فشاری، LED، بیزر و ترازیستور
	\subsection{شرح آزمایش}
	
	\subsubsection{سرکشی پایه های مختلف}	
		در قسمت اول آزمایش قصد روشن و خاموش کردن یک \lr{LED} را داریم که برای اینکار نیاز به اطلاع از وضعیت پایه \lr{PD2} داریم که برای اینکار از روش سرکشی استفاده می کنیم
		
		برای روشن کردن بیزر نیز روش مشابهی را در پیش می گیریم در اینجا با \lr{High} کردن پایه \lr{PC0} ترانزیستور روشن و در انتها بیزر به صدا در می آید
		
		برای وصل کردن رله نیز اقدامی مشابه بالا صورت میگیرد (پایه \lr{PC1})
		\\
		برای کار کردن با ورودی ها میتوان از 3 ثبات مخصوص هر ورودی
		داریم 
		
	\subsubsection{تاخیر نرم افزاری}
		برای اینکار باید به صورت  نرم افزاری و بدون استفاده از زمان سنج تاخیر ایجاد کنیم
		
		\noindent		
		برای مثال میتوان با استفاده از دستوری که تکرار می شود مثل یک حلقه، تاخیری بر مبنای کلاک میکرو ایجاد کرد
		
		\vspace{10mm}
		\begin{latin}
		\noindent ldi r16,123 \\
		DELAY: \\
		dec r16 \\
		brne DELAY \\
		\end{latin}
	
		میتوان از دستور \lr{nop} برای ایجاد تاخیر بیشتر استفاده نمود
		
	\subsubsection{زمان سنج نگهبان}
		زمان سنج نگهبان \lr{watchdog timer} که برای بازنشانی میکرو به کار می رود که در بازنشانی میکرو همه ثبات ها را به مقدار اولیه خود برمیگرداند و در مواردی مانند 
		\lr{hang} کردن میکرو به کار می رود در واقع اگر برنامه ما به درستی اجرا شود باید در بازه های زمانی مشخصی زمان سنج نگهبان را ریست کند اما اگر میکرو هنگ کرده 
		باشد برنامه زمان سنج را ریست نمی کند در نتیجه با صفر شدن مقدار آن میکرو بازنشانی(ریست) می شود و به وضعیت پایدار برمیگردد
		
\end{persian}
\end{document}